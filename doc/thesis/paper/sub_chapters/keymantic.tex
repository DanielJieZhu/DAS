
\cite{Keymantic10, semantics_without_access} explores the case then there is no possibility to index the data terms, e.g. then a DB is behind a wrapper (e.g. accessible only through a \textit{Web form} in “Hidden Web” or \textit{a web-service}) then crawling is generally not possible.
%
In Keymantic\cite{Keymantic10} a keyword query is processed as follows: First, all keywords that  correspond to metadata items (e.g., field names) are extracted. The remaining keywords are considered as possible parameters to the input fields in the web form. Second, the likelihood of a remaining keyword to matching a metadata item is computed in order to rank different options of executing this keyword query on the "Hidden Web"\cite[p.942]{ethz2012}. Because of no access to the actual data, results of this method were reported considerably worse on queries containing data terms, even if all metadata (e.g. business ontology) is given\cite{ethz2012}, therefore it is helpful to have at least some insight on the data behind a webservice.
% See Appendix #1, for evaluation of they Demo system.
