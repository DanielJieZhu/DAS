\section{Preliminary Literature review}

% Overview: keyword search is good for non-structured documents, it is not [as] effective with structured sources\cite{Levy96}, therefore the keyword search on relational databases approach is good then no specific APIs exist and it's there are no resources to do \textbf{a full integration} .

\subsection{Overview}


The problem of keyword search over \textit{relational and other structured databases} is widely covered,  including even keyword search directly over distributed relational databases\cite[ch.16]{principles_data_integration} and some approaches for automatic schema matching has been discussed also. Such search is however very ambiguous and has quite large search space over the possible join paths, therefore most approaches rely on returning the top-k results.

Regarding the Boolean retrieval that we are the most interested, keyword query could be translated into list of best structured queries (e.g. SODA system proposing SQL over complex schemas, while Query Forms would propose mappings from keywords to parameters in SQL templates, Keymantic tries to achieve this without having access to Data itself). 
% TODO: Keymantic would propose SQL queries knowing only schema but no index

%  TODO: Further, {\color{red} in theory,} Google's Deep-web crawl techniques initially designed for accessing content hidden behind html forms could be used to effectively index the contents even of  data providers that do not provide a specific interface to do so.

{\color{red}Current research trends seem to fall on methods of automatic integration (e.g. statistical mediator), and pay-as-you-go integration. (TODO: see future chapter of \cite{principles_data_integration})}

%\textbf{\color{red}TODO: Overview; evaluate performance}

\subsection{Searching 'Deep Web' and web services}

In 1990s there were was much research on search based service integration systems, for instance \textit{Information Manifold} quite resembling DAS {\color{red},  so it could be useful checking it's papers.}.
% following Local-as-View approach
%and \textit{TSIMMIS} developed at Stanford following Global-as-View approach. The \textit{Information Manifold} is 

\subsubsection*{No access to index data terms}

\cite{Keymantic10, semantics_without_access} explores the case then there is no possibility to index the data terms, e.g. then a DB is behind a wrapper (e.g. accessible only through a \textit{Web form} in “Hidden Web” or \textit{a web-service}) then crawling is generally not possible.

In Keymantic\cite{Keymantic10} a 'keyword query is processed as follows: First, all keywords that correspond to metadata items (e.g., field names) are extracted. The remaining
keywords are considered as possible input fields. Second, the likelihood of a remaining keyword to a metadata item is computed in order to rank different options to execute the keyword query on the
“Hidden Web” database'\cite[p.942]{ethz2012}.


\subsubsection*{Deep web search at Google}
[multiple papers from Google] discusses various ways for implementing data integration in terms of large-scale search engine (Google): Virtual integration vs. Surfacing. They also present ways for integrating systems without human intervention through use of statistical 'mediator'.

There two approaches to web scale search for deep-web (Google mostly cares about web forms):

\textit{Runtime query reformulation} - 'leaves data at the sources and routes queries to appropriate services'\cite[p. 1]{webscale_paygo} 

\textit{Deep-web surfacing} - tries to add content from the deep-web into search index. There are algorithms which allow to iteratively choose input parameters to forms to surface a considerable part of the 'hidden' data without large overhead%
	\footnote{i.e. if choosing parameters in not smart way, a web form with just a couple of free inputs (or even dropdowns that's easier case), could yield as many results as a cross product of all input combinations.}.

\textit{PayGo approach}\cite{webscale_paygo}  - With this approach, 'a system starts with
very few (or inaccurate) semantic mappings and these mappings
are improved over time as deemed necessary'.
% TODO: cite{bootstraping}
there is NO single mediated schema over which users pose queries. Queries are routed to to the relevant sources with help statistical methods that are used to model uncertainty at all levels: queries, mappings and underlying data. 

%{\color{red} TODO: Describe Keymantic\cite{Keymantic10}. This is a very suboptimal solution as indexing terms improves results (mention evaluation from SODA paper). It works only then keywords map entity names. some hybrid approaches: 
%index if exists, regexp, some string similarity measure based on historical data (e.g. even edit-distance would work for many items, like site)}



\subsection{Keyword search over relational Database}
The problem of Keyword search over Relational Databases (or also semi-structured sources like XML) has received a significant attention by the research community over the last decade. 

The basic approach would first build an inverted index on database tables (usually only text columns). Then after finding all occurrences of the keywords, would try to construct join paths (based on Foreign keys) that would unite tuples containing the keywords.

% In addition to many papers the PhD dissertations \cite{PhD_2011, PhD_2012} describe the approaches in details including performance optimization details (e.g. generating materialized views), etc. 
% 
A number of problem variations exist:  returning only the ranked Top-k results vs. returning ranked list of possible queries, while some systems would even allow generating more complex queries including aggregations, etc (SQAK, SODA\cite{ethz2012}).

\subsubsection*{Ranking Query Templates based on keyword query}
A simple way to access relational database could  be through a set of predefined named query templates (SQL with selection parameters or operators still to be specified) exposed to a user as a Form that the user has to fill in.

\cite{forms_kws} proposes alternative approach for processing keyword queries over relational databases: given a keyword query, instead of returning database tuples one could rank query forms that best matches the query for user to choose the right one (if they are properly named this is fairly easy). The ranking is based on checking matching of keywords to table names in templates and to column values (could be implemented with inverted index).\textbf{\color{red}TODO: more detailed and our limitations (after reading keyword cleaning)}.

An interesting feature of this approach is that a Query Template is functionally similar to any autonomous web service (which given the parameters would in turn execute that query on its database).  In case of the Data Aggregation System, a user after entering a keyword query could be provided with a ranked list of structured queries (attribute=value) that could be processed given data source constraints (e.g. parameters required) and if needed he could refine his search (e.g. provide more parameters).


\subsubsection*{Keyword query cleaning}
Keyword queries are often ambiguous, may contain misspellings or multiple keywords that refer to the same attribute value,  therefore \cite{kw_cleaning} suggested to perform query cleaning before proceeding to subsequent more computationally expensive steps (e.g. exploring all the possible join paths).

Further employing some machine learning method like HMM\cite{kw_cleaning_hmm} would allow to incorporate user's feedback (even the fact that user has chosen n-th result as a query to be executed is a good clue).


\subsubsection*{SODA: Meta-data approach}

With a goal to bridge the increasing gap between high-level (conceptual, business) and low level (physical) representations of data, researchers from \textit{ETHZ} have been investigating Generation of SQL for Business users over a very complex data warehouse at \textit{Credit Suisse}.  For converting natural language queries 
% (that in addition to keyword search could convey some semantic structure)
 into SQL statements, in addition to what used by earlier approaches they used meta-data describing the schema at both physical, conceptual and logical levels extended with DBpedia (for synonyms, etc) and domain ontologies (to capture business concepts like 'wealthy customer').
%and some natural language processing

Even on a large data-warehouse of ~220GB data with a complex schema of 400+ tables they reported that if good meta-data is available, generating even fairly complex SQL  (e.g. n-way joins with aggregations) is quite feasible for a computer. That would make it 'much easier for business users to interactively explore highly-complex data warehouses' \cite[p.932]{ethz2012}. The users also reported system's potential a) for analysing the schema and learning patterns about it and b) as tool to help documenting legacy systems.


\subsection{Keyword search: integration on demand}
{\color{red}TODO?: \cite[ch.16]{principles_data_integration}}




%\subsubsection*{Automatic mapping between distributed services}
%{\color{red} TODO: an interesting approach but not for CERN...}



%\subsubsection*{\color{red}Question Answering}


% See Appendix #1, for evaluation of they Demo system.



% {\color{red}There doesn't seem to be much of recent papers going towards this direction (except from specific search systems like flight fare comparisons), however these are based on structured arguments.} 

% Some other approaches could include: - creating virtual documents offline by joining tables, for instantaneous search results by employing IR techniques
% (Indexing Relational Database Content Offline for Efficient Keyword-Based Search, 2003)

% TODO: (Efficient Keyword Search Across Heterogeneous Relational Databases, 2007) : combines schema matching and structure discovery techniques to find approximate foreign-key joins across heterogeneous databases

