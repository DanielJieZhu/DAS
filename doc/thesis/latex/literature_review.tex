\section{Preliminary Literature review}

% Overview: keyword search is good for non-structured documents, it is not [as] effective with structured sources\cite{Levy96}, therefore the keyword search on relational databases approach is good then no specific APIs exist and it's there are no resources to do \textbf{a full integration} .

\textbf{\color{red}TODO: Overview; evaluate performance}

\subsection{Searching 'Deep Web' and Heterogeneous web services}

In 1990s there were was much research on search based service integration systems, for instance \textit{Information Manifold} following Local-as-View approach and \textit{TSIMMIS} developed at Stanford following Global-as-View approach. The \textit{Information Manifold} is quite resembling to DAS {\color{red},  so it could be useful checking it's papers. TODO: finish}


\subsubsection*{Deep web search at Google}
[multiple papers from Google] discusses various ways for implementing data integration in terms of large-scale search engine (Google) Virtual integration vs. surfacing. They also present ways for integrating systems without human intervention through use of statistical 'mediator'.

There two approaches to web scale search for deep-web (Google mostly cares about web forms):

\textit{Runtime query reformulation} - 'leaves data at the sources and routes queries to appropriate services'\cite[p. 1]{paygo_integration} 

\textit{Deep-web surfacing} - tries to add content from the deep-web into search index. There are algorithms which allow to iteratively choose input parameters to forms to surface most of the 'hidden' data without loosing much (i.e. if choosing parameters in not smart way the web form could yield as many results as a cross product of all input combinations).

PayGo approach - there is NO single mediated schema over which users pose queries. Queries are routed to to the relevant sources. Statistical methods used to model uncertainty at all levels: queries, mappings and underlying data.


\subsubsection*{Then there is no access to index data terms}
{\color{red} TODO: Describe Keymantic\cite{Keymantic10}. This is a very suboptimal solution as indexing terms improves results (mention evaluation from SODA paper). It works only then keywords map entity names.

some hybrid approaches: 
index if exists
regexp
some string similarity measure based on historical data (e.g. even edit-distance would work for many items, like site)}



\subsection{Keyword search over DBs}
The problem of Keyword search over Relational Databases (or also semi-structured sources like XML) has received a significant attention by the research community over the last decade. 

The basic approach would first build an inverted index on database tables (usually only text columns). Then after finding all occurrences of the keywords, would try to construct join paths (based on Foreign keys) that would unite tuples containing the keywords.

% In addition to many papers the PhD dissertations \cite{PhD_2011, PhD_2012} describe the approaches in details including performance optimization details (e.g. generating materialized views), etc. 
% 
A number of problem variations exist:  returning only the ranked Top-k results vs. returning ranked list of possible queries, while some systems would even allow generating more complex queries including aggregations, etc (SQAK, SODA\cite{ethz2012}).

\subsubsection*{Ranking Query Templates based on keyword query}
A simple way to access relational database could  be through a set of predefined named query templates (SQL with selection parameters or operators still to be specified) exposed to a user as a Form that the user has to fill in.

\cite{forms_kws} proposes alternative approach for processing keyword queries over relational databases: given a keyword query, instead of returning database tuples one could rank query forms that best matches the query for user to choose the right one (if they are properly named this is fairly easy). The ranking is based on checking matching of keywords to table names in templates and to column values (could be implemented with inverted index).\textbf{\color{red}TODO: more detailed and our limitations (after reading keyword cleaning)}.

An interesting feature of this approach is that a Query Template is functionally similar to any autonomous web service (which given the parameters would in turn execute that query on its database).  In case of the Data Aggregation System, a user after entering a keyword query could be provided with a ranked list of structured queries (attribute=value) that could be processed given data source constraints (e.g. parameters required) and if needed he could refine his search (e.g. provide more parameters).


\subsubsection*{Keyword query cleaning}
Keyword queries are often ambiguous, may contain misspellings or multiple keywords that refer to the same attribute value,  therefore \cite{kw_cleaning} suggested to perform query cleaning before proceeding to subsequent more computationally expensive steps (e.g. exploring all the possible join paths).

Further employing some machine learning method like HMM\cite{kw_cleaning_hmm} would allow to incorporate user's feedback (even the fact that user has chosen n-th result as a query to be executed is a good clue).


\subsubsection*{Meta-data approach}

With a goal to bridge the increasing gap between high-level (conceptual, business) and low level (physical) representations of data, researchers from \textit{ETHZ} have been investigating Generation of SQL for Business users at \textit{Credit Suisse}.  For converting natural language queries 
% (that in addition to keyword search could convey some semantic structure)
 into SQL statements, in addition to what used by earlier approaches they used meta-data describing the schema (at multiple representation levels) and the domain (ontologies) and some natural language processing.

Even on a large data-warehouse of ~220GB data with a complex schema of 400+ tables they reported that if good meta-data is available, generating even quite complex SQL  (n-way joins with aggregations, etc) is quite feasible for computer. That would making it 'much easier for business users to interactively explore highly-complex data warehouses' \cite[p.932]{ethz2012}. 




%\subsubsection*{Automatic mapping between distributed services}
%{\color{red} TODO: an interesting approach but not for CERN...}



%\subsubsection*{\color{red}Question Answering}


% See Appendix #1, for evaluation of they Demo system.



% {\color{red}There doesn't seem to be much of recent papers going towards this direction (except from specific search systems like flight fare comparisons), however these are based on structured arguments.} 

% Some other approaches could include: - creating virtual documents offline by joining tables, for instantaneous search results by employing IR techniques
% (Indexing Relational Database Content Offline for Efficient Keyword-Based Search, 2003)

% TODO: (Efficient Keyword Search Across Heterogeneous Relational Databases, 2007) : combines schema matching and structure discovery techniques to find approximate foreign-key joins across heterogeneous databases

